\documentclass{article}

\usepackage[utf8]{inputenc}
\usepackage{tikz}
\usepackage[italian]{babel}
\usepackage{amsfonts}
\usepackage{amssymb}
\usepackage[export]{adjustbox}
\usepackage[utf8]{inputenc}
\usepackage[T1]{fontenc}
\usepackage{graphicx}
\usepackage{titlepic}
\usepackage{fixltx2e}
\usepackage{mathrsfs}  
\usepackage{mathtools}
\usepackage{lipsum}
\usepackage{soul}
\usepackage{amsmath}
\usepackage{amsthm}
\usepackage{tabu}
\usepackage{booktabs}
\usepackage{float}
\usepackage{subfig}
\usepackage{ulem}
\usepackage{accents}
\usepackage{multicol}
\usepackage[hidelinks]{hyperref}
\usepackage{setspace}
\usepackage{enumitem}
\usepackage{listings}
\usepackage{mathtools}
\usepackage{fixmath}
\usepackage{sectsty} 
\usepackage[margin=30mm, bottom=32mm]{geometry} 
\usepackage{geometry} 
\usepackage{changepage} 
\usepackage[font=footnotesize,labelfont=bf, skip=4pt]{caption} 
\usepackage{makecell}
\usepackage{textgreek}
\usepackage{booktabs}
\usepackage{multirow}
\usepackage{siunitx}

\begin{document}

\begin{titlepage}
    \begin{center}
        \Large
        Università degli Studi di Trento

        \vspace{0.5cm}

        \Large
        Introduzione alla Programmazione per il web

        \includegraphics[width=0.6\textwidth]{logo_home_new.png}
        
        \vfill

        \Huge
        \textbf{Procedura di testing}

        \vspace{0.5cm}

        \Large
        Strumento 1 - SUS

        \vspace{5cm}
        
        \Large
        \textbf{Gruppo 19}

        \vspace{0.5cm}
        
        \Large
        Anno accademico 2022/23
    \end{center}
\end{titlepage}

\tableofcontents
\pagebreak

\noindent Di seguito viene presentata una procedura di testing finalizzata a valutare il sito web sviluppato per l'associazione \textit{Tum4World} in relazione a diversi criteri di usabilità. In particolare, vengono presi in considerazione: l'apprendimento rapido, la necessità di supporto tecnico, l'organizzazione del sito, la facilità d'uso, l'integrazione delle funzionalità, la coerenza tra le pagine, la necessità di studio preliminare del sito e il comfort durante l'utilizzo.\\

\noindent La procedura di testing coinvolge un gruppo di 100 partecipanti selezionati in base all'età e al profilo professionale, in modo da ottenere un campione di utenti il più possibile omogeneo e garantire quindi la massima affidabilità dei risultati del test. I partecipanti selezionati non hanno familiarità precedente con il sito.

\vspace{0.7cm}

\section{Valutazione dell'apprendimento rapido del sito}
Per valutare l'apprendimento rapido del sito, è stata scelta una specifica sequenza di operazioni.\\
Ai partecipanti viene richiesto di condividere un contatto per consentire all'associazione di contattarli in seguito. Il contatto da utilizzare viene indicato dal team di testing:
\begin{itemize}
    \item nome: Paolino
    \item cognome: Paperino
    \item indirizzo email: paolino.paperino@gmail.com
\end{itemize}
Ciascun partecipante ha a disposizione 3 minuti per portare a termine il compito assegnato. Il test è considerato superato se almeno 70 partecipanti sono riusciti a effettuare tutte le operazioni entro il tempo stabilito.\\
Si noti che ai partecipanti non vengono fornite ulteriori istruzioni oltre a quelle menzionate. È quindi compito del partecipante individuare autonomamente quale pagina selezionare e quali operazioni eseguire.

\vspace{0.7cm}

\section{Valutazione della necessità di supporto tecnico}
La necessità di supporto tecnico viene valutata attraverso il monitoraggio delle richieste di assistenza effettuate dai partecipanti durante lo svolgimento dell'intero test. Il supporto tecnico è previsto per tutte le fasi della procedura di testing, tranne per la valutazione dell'apprendimento rapido.\\
Il test è considerato superato se, in media, le richieste di assistenza da parte dei partecipanti sono inferiori a 3 per partecipante.

\vspace{0.7cm}

\section{Valutazione della complessità dell'organizzazione del sito}
Per valutare l'organizzazione del sito, ai partecipanti è richiesto di individuare una serie di informazioni specifiche all'interno del sito:
\begin{itemize}
    \item anno di fondazione;
    \item nomi delle attività offerte;
    \item indirizzo email e numero di telefono.
\end{itemize}
Ciascun partecipante ha a disposizione 5 minuti per portare a termine il compito assegnato. Il test è considerato superato se almeno 90 partecipanti sono riusciti a trovare tutte le informazioni richieste entro il tempo stabilito.\\
In seguito, ogni partecipante assegna un voto compreso tra 1 e 5 considerando la seguente scala di valutazione:
\begin{enumerate}
    \item l'organizzazione del sito è molto complessa e individuare le informazioni ricercate risulta difficile;
    \item l'organizzazione del sito è complessa, ciò nonostante le informazioni ricercate si individuano con poche difficoltà;
    \item il sito è ben organizzato, ma le informazioni ricercate non sono facili da individuare;
    \item il sito è organizzato molto bene e le informazioni ricercate si individuano facilmente;
    \item l'organizzazione del sito è ottima e individuare le informazioni ricercate risulta estremamente semplice.
\end{enumerate}
L'organizzazione del sito è considerata buona e il test superato se la somma dei voti dei partecipanti è superiore a 350.

\vspace{0.7cm}

\section{Test di usabilità del sito}
Per valutare l'usabilità del sito, i partecipanti sono invitati a completare una serie di compiti che coinvolgono le principali funzionalità del sito.\\
Il team di testing mette a disposizione dei partecipanti i seguenti dati:
\begin{itemize}
    \item nome: Paperon
    \item cognome: de Paperoni
    \item data di nascita: 15/12/1947
    \item indirizzo email: paperon.depaperoni@gmail.com
    \item numero di telefono: 0123456789
    \item tipo di utente: aderente
    \item username: zio\_Paperone
    \item password: 3EFGroni!
\end{itemize}
A ciascun partecipante è richiesto di effettuare le seguenti operazioni considerando i dati suddetti:
\begin{itemize}
    \item effettuare il sign in;
    \item effettuare il login;
    \item visualizzare i dati personali;
    \item iscriversi a una delle attività offerte;
    \item effettuare una donazione di 50 Euro (operazione fittizia);
    \item cancellare la propria iscrizione al sito.
\end{itemize}
Ogni partecipante ha a disposizione 15 minuti per portare a termine il compito assegnato. Il test è considerato superato se almeno 70 partecipanti sono riusciti a compiere tutte le operazioni richieste entro il tempo stabilito.\\
Inoltre, ai partecipanti è richiesto di effettuare il sign in utilizzando anche i seguenti dati:
\begin{itemize}
    \item nome: John
    \item cognome: Rockerduck
    \item data di nascita: 11/12/2011
    \item indirizzo email: john.rockerduckgmail.com
    \item numero di telefono: 01234567
    \item tipo di utente: aderente
    \item username: Rockerduck
    \item password: 7EFGduck!
\end{itemize}
Si noti che la data di nascita, l'indirizzo email e il numero di telefono non sono validi. Pertanto, i partecipanti sono invitati a correggere i dati inseriti in modo da consentire l'invio del form.\\
Ogni partecipante ha a disposizione 10 minuti per portare a termine il compito assegnato. Il test è considerato superato se almeno 70 partecipanti riusciranno a inviare il form di sign in entro il tempo stabilito.\\
Infine, viene considerato un modulo di feedback che presenta le seguenti domande:
\begin{itemize}
    \item Effettuare le operazioni risulta semplice e gli errori segnalati dal sito sono chiari ed efficaci?
    \item Le informazioni inserite nella pagina sign in per facilitare la compilazione del form sono comprensibili?
    \item I dati personali visualizzati corrispondono a quelli inseriti durante il sign in?
    \item Cancellare la propria iscrizione al sito risulta semplice e intuitivo? 
    \item L'iscrizione alle attività offerte risulta facile e veloce?
\end{itemize}
Per ciascuna domanda, si può rispondere \textit{sì} oppure \textit{no}. Se il numero di \textit{sì} supera il 70 \% del totale delle risposte, allora il feedback è considerato positivo e il test di usabilità del sito superato.

\vspace{0.7cm}

\section{Valutazione dell'integrazione delle funzionalità del sito}
Per valutare l'integrazione delle funzionalità del sito, è stato scelto uno specifico flusso di lavoro.\\
Vengono messi a disposizione dei partecipanti lo username e la password di un amministratore:
\begin{itemize}
     \item username: admin
    \item password: 19Adm1n!
\end{itemize}
A ciascun partecipante è richiesto di inserire nella pagina di login lo username e la password dell'amministratore, ed effettuare le seguenti operazioni:
\begin{itemize}
    \item visualizzare l'elenco di tutti gli utenti;
    \item visualizzare l'elenco dei simpatizzanti;
    \item visualizzare l'elenco degli aderenti;
    \item visualizzare il numero di visite del sito e l'istogramma delle visite per ogni pagina;
    \item resettare i contatori;
    \item visualizzare il grafico delle donazioni.
\end{itemize}
Il tempo messo a disposizione di ogni partecipante per portare a termine il compito assegnato è di 15 minuti. È stato scelto un intervallo di tempo ampio per consentire a tutti i partecipanti di eseguire tutte le operazioni richieste senza sentirsi limitati dal tempo.\\
In seguito, viene fatta la seguente domanda: "Esistono delle funzionalità che svolgono in parte la stessa funzione?". Nel caso di risposta affermativa, si richiede di specificare le funzionalità coinvolte. Se il numero di \textit{no} supera il 70 \% del totale delle risposte, allora il test di integrazione delle funzionalità del sito è considerato superato.

\vspace{0.7cm}

\section{Analisi della coerenza tra le pagine del sito}
La coerenza tra le pagine del sito viene valutata attraverso un'analisi dettagliata delle varie sezioni. I partecipanti sono invitati a navigare tra le pagine e a individuare eventuali difformità e incongruenze nell'aspetto visivo e nella disposizione dei contenuti, e possibili inconsistenze tra le informazioni fornite.\\
In particolare, si considera un modulo di feedback con le seguenti domande:
\begin{itemize}
    \item Le pagine che forniscono funzionalità simili utilizzano un layout uniforme?
    \item L'utilizzo dei colori consente un'esperienza coerente e intuitiva?
    \item La disposizione dei contenuti e l'aspetto visivo del sito risultano armoniosi?
    \item Il tipo di carattere utilizzato è uniforme in tutte le pagine?
    \item Le informazioni reperibili dalle varie pagine sono tutte consistenti tra loro?
\end{itemize}
Per ciascuna domanda, si può rispondere \textit{sì} oppure \textit{no}. Se il numero di \textit{sì} supera il 70 \% del totale delle risposte, allora il feedback è considerato positivo e il test della coerenza tra le pagine del sito superato.

\vspace{0.7cm}

\section{Valutazione della necessità di studio preliminare del sito}
La necessità di studio preliminare del sito è valutato attraverso un modulo di feedback che presenta le seguenti domande:
\begin{itemize}
    \item La navigazione tra le pagine risulta molto semplice?
    \item Le funzionalità del sito sono facili da utilizzare e intuitive?
    \item Il sito è accessibile anche a persone con minori competenze nell'uso del computer?
    \item I menu di navigazione sono chiari ed esplicativi?
    \item Il sito ha soddisfatto le aspettative in termini di facilità d'uso?
    \item Il linguaggio utilizzato nel sito è comprensibile per utenti con diversi livelli di competenze?
    \item Il layout è intuitivo e facilita la navigazione tra le pagine?
\end{itemize}
Queste domande hanno l'obbiettivo di misurare il livello di intuitività e accessibilità del sito.\\
Prima di rispondere alle domande proposte con \textit{sì} o \textit{no}, i partecipanti devono aver seguito la procedura di testing fino al punto 6. Se il numero di \textit{sì} supera il 70 \% del totale delle risposte, allora il feedback è considerato positivo e il sito non necessita di uno studio preliminare per essere utilizzato in modo efficiente.

\vspace{0.7cm}

\section{Valutazione del comfort durante l'utilizzo del sito}
Il comfort durante l'utilizzo del sito è valutato attraverso un modulo di feedback che presenta le seguenti domande:
\begin{itemize}
    \item L'esperienza complessiva di utilizzo del sito è soddisfacente?
    \item L'aspetto visivo e il design del sito sono esteticamente piacevoli?
    \item L'utilizzo dei colori è gradevole alla vista?
    \item Le informazioni sono facilmente reperibili e organizzate in modo chiaro e intuitivo?
    \item Il sito risponde alle aspettative degli utenti in termini di funzionalità?
    \item Il tempo di risposta del sito è rapido, garantendo un'esperienza fluida e senza attese?
    \item Le funzionalità del sito sono comprensibili e facilmente utilizzabili?
\end{itemize}
Prima di rispondere alle domande proposte con \textit{sì} o \textit{no}, i partecipanti devono aver seguito la procedura di testing fino al punto 6. Se il numero di \textit{sì} supera il 70 \% del totale delle risposte, allora il feedback è considerato positivo e il test del comfort durante l'utilizzo del sito è considerato superato.

\end{document}
